\documentclass[11pt, a4paper]{article}
\usepackage{subfiles}

\input{../macro/packages.tex}
\input{../macro/settings.tex}
\input{../macro/new_commands.tex}


\begin{document}

\author{Rocco Ardino\\1231629  \and Alessandro Lambertini\\ 1242885 \and Alice Pagano \\1236916 \and Michele Puppin \\ 1227474}
\title{\textbf{Management and analysis of physics dataset: \\ FPGA  Stopwatch (modulo 16)}}
\maketitle

\section{Aim}
The purpose of the assignment is to implement a 4-bit stopwatch, namely a counter, with the following functionalities:
\begin{itemize}
\item \textbf{START}: it enables the counting.
\item \textbf{STOP}: it stops the counting.
\item \textbf{RESET}: it resets the counting.
\item \textbf{FREQUENCY SELECTOR}: it can change the frequency of the counting.
\item \textbf{REVERSE SELECTOR}: it makes the stopwatch counting in reverse.
\end{itemize}





\section{Implementation}
The Arty7 board has 4 LEDs that could be used as a display counter \( (0 \rightarrow 15) \). Indeed, each LED is associated to a bit: an off LED corresponds to the bit state '0', while a blinking one corresponds to the bit state '1'. The time flow is regulated by the embedded clock of the board, which is used to implement the counter.
Concerning the functionalities, START, STOP and RESET can be triggered by the embedded buttons of the board, while FREQUENCY and REVERSE SELECTORs by the four switches. In particular, three switches are used for modulating the frequency of the counting and one is used for reversing it. The actual disposition of functions is illustrated in Figure \ref{fig:board_implementation}.

\begin{figure}[H]
\centering
\includegraphics[width=0.7\textwidth]{../main/image/board_implementation.pdf}
\caption{\label{fig:board_implementation} Arty7 board: disposition of the functionalities.}
\end{figure}


\clearpage
\subsection{Counter}
The code implementation is constituted by four main processes. The first two processes ({\footnotesize\texttt{p\_cnt}} and {\footnotesize\texttt{p\_slw\_cnt}}) are used to implement the counter.

First of all the clock has been used to increase the value of a vector of 28 elements ({\footnotesize\texttt{counter}}) each time the clock signal shows a rising edge, as in process {\footnotesize\texttt{p\_cnt}} (Listing \ref{lst:p_cnt}).

\begin{lstlisting}[style=vhdl,label={lst:p_cnt},caption={\texttt{p\_cnt} process.}]
p_cnt : process(clk,rst,sel_in) is
    begin
    if rst = '1' then
        counter <= (others => '0');
    end if;
    if rising_edge(clk) then
        counter <= counter +1;
    end if;
end process;\end{lstlisting}

Then, since the speed of the embedded clock is too fast, it has been slowed down in process {\footnotesize\texttt{p\_swl\_cnt}} (Listing \ref{lst:p_slw_cnt}) by creating a slow clock signal {\footnotesize\texttt{slow\_clk}} and taking the \(i^\text{th}\) bit {\footnotesize\texttt{slow\_clk <= counter(i)}}, where the value of \(i\) is determined by the frequency selector.
The slow clock signal is eventually been used to update the {\footnotesize\texttt{slow\_counter}} signal which reflects the four bit display counter that will be mapped to the LEDs.

\begin{lstlisting}[style=vhdl,label={lst:p_slw_cnt},caption={\texttt{p\_slw\_cnt} process.}]
p_slw_cnt : process(clk,rst,frz,slow_clk,sel_in,state) is
    begin
    if rst = '1' then
        slow_counter <= (others => '0');
    end if;
    if rising_edge(clk) then
        slow_clk_p <= slow_clk;
        if state = '0' then
            if slow_clk = '1' and slow_clk_p = '0' then
                if sel_in(0) = '0' then
                    slow_counter <= slow_counter + 1;
                elsif sel_in(0) = '1' then
                    slow_counter <= slow_counter - 1;
                end if;
            end if;
        end if;
    end if;
end process;\end{lstlisting}

\subsection{FREQUENCY and REVERSE SELECTOR}
In order to select manually the speed of the display counter, the value \(i^\text{th}\) element of the vector {\footnotesize\texttt{counter}} is chosen based on the values of the three elements of {\footnotesize\texttt{sel}} vector that reflect the position of three switches. It is implemented in the {\footnotesize\texttt{speed}} process (Listing \ref{lst:speed}).
Morover, in order to reverse the counter, {\footnotesize\texttt{slow\_counter}} is updated forward \( (+1)\) or backward \((-1)\) based on the position of a fourth switch as shown in process {\footnotesize\texttt{p\_slw\_cnt}} (Listing \ref{lst:p_slw_cnt}).

\begin{lstlisting}[style=vhdl,label={lst:speed},caption={\texttt{speed} process.}]
speed : process(clk,rst,slow_clk,sel) is
    begin
    case sel is
        when "000" => slow_clk <= counter(27);
        when "001" => slow_clk <= counter(26);
        when "010" => slow_clk <= counter(25);
        when "011" => slow_clk <= counter(24);
        when "100" => slow_clk <= counter(23);
        when "101" => slow_clk <= counter(22);
        when "110" => slow_clk <= counter(21);
        when "111" => slow_clk <= counter(20);
        when others => null;
    end case;
end process;\end{lstlisting}

\subsection{START, STOP and RESET}
START and STOP button has been implemented in the process {\footnotesize\texttt{btn\_state}} (Listing \ref{lst:btn_state}), where a variable {\footnotesize\texttt{state}} is set to 1 if STOP button is pressed while it is st to '0' if START is pressed.
In process {\footnotesize\texttt{p\_slw\_cnt}}, previously described in Listing \ref{lst:p_slw_cnt}, the update of {\footnotesize\texttt{slow\_counter}} happens only if {\footnotesize\texttt{state}} is set to '0'. Instead, RESET button, besides setting {\footnotesize\texttt{state}} to '1' in order to stop the counter, sets {\footnotesize\texttt{counter}} and {\footnotesize\texttt{slow\_counter}} to '0'.

\begin{lstlisting}[style=vhdl,label={lst:btn_state},caption={\texttt{btn\_state} process.}]
btn_state : process(clk,rst,frz,start,slow_clk,sel_in,state) is
    begin
    if rising_edge(clk) then
        if start = '1' then
            state <= '0';
        elsif frz = '1' then
            state <= '1';
        elsif rst = '1' then
            state <= '1';
        end if;
    end if;
end process;\end{lstlisting}

\section{Simulation}

\begin{figure}[H]
\centering
\includegraphics[width=1\textwidth]{../main/image/sim_2.png}
\caption{\label{fig:sim_1} Simulation reset stop.}
\end{figure}







\end{document}
